% Uncomment the following to produce a twoside document, suitable for printing.
\usepackage[inner=1.5in,outer=1in,letterpaper,twoside]{geometry}

%%%%%%%%%%%%%%%%%%%%%%%%%%%%%%%%%%%%%%%%%%%%%%%%%%%%%%%%%%%%%%%%%%%%%%%%%%%%%%
% Packages

% The 'graphicx' package is quite extensive; the main purpose here is
% the includion of PostScript graphics.  The 'dvips' option makes it work
% better with 'dvips' (which is what is often used under Linux)
%\usepackage[dvips]{graphicx}
% Comment the above line and uncomment below to use pdflatex
\usepackage{graphicx}
% Then the figures should be in pdf, jpg or png format.

% Others to consider
\usepackage{listings}
\usepackage{color}
\definecolor{listinggray}{gray}{0.95}
\definecolor{mygreen}{rgb}{0,0.6,0}
\definecolor{mygray}{rgb}{0.5,0.5,0.5}
\definecolor{mymauve}{rgb}{0.58,0,0.82}
\definecolor{maroon}{rgb}{0.5,0.00,0.0}


\lstdefinestyle{javanum}{
  backgroundcolor=\color{listinggray},   % choose the background color; you must add \usepackage{color} or \usepackage{xcolor}
  basicstyle=\footnotesize\ttfamily, % the size of the fonts that are used for the code
  breakatwhitespace=false,         % sets if automatic breaks should only happen at whitespace
  breaklines=true,                 % sets automatic line breaking
  captionpos=none,                   % sets the caption-position to bottom
  aboveskip=\smallskipamount,
  belowskip=\smallskipamount,     % default is \medskipamount
  commentstyle=\color{mygreen},    % comment style
  deletekeywords={...},            % if you want to delete keywords from the given language
  escapeinside={\%*}{*)},          % if you want to add LaTeX within your code
  extendedchars=true,              % lets you use non-ASCII characters; for 8-bits encodings only, does not work with UTF-8
  frame=none,                      % adds a frame around the code (none, single)
  keepspaces=true,                 % keeps spaces in text, useful for keeping indentation of code (possibly needs columns=flexible)
  keywordstyle=\color{maroon},       % keyword style
  language=Java,                   % the language of the code
  morekeywords={*,...},            % if you want to add more keywords to the set
  numbers=left,                    % where to put the line-numbers; possible values are (none, left, right)
  numbersep=5pt,                   % how far the line-numbers are from the code
  numberstyle=\tiny\color{mygray}, % the style that is used for the line-numbers
  rulecolor=\color{black},         % if not set, the frame-color may be changed on line-breaks within not-black text (e.g. comments (green here))
  showspaces=false,                % show spaces everywhere adding particular underscores; it overrides 'showstringspaces'
  showstringspaces=false,          % underline spaces within strings only
  showtabs=false,                  % show tabs within strings adding particular underscores
  stepnumber=1,                    % the step between two line-numbers. If it's 1, each line will be numbered
  stringstyle=\color{mymauve},     % string literal style
  tabsize=4,                       % sets default tabsize to 2 spaces
%   title=\lstname                   % show the filename of files included with \lstinputlisting; also try caption instead of title
}

\lstdefinestyle{javanonum}{
  backgroundcolor=\color{listinggray},   % choose the background color; you must add \usepackage{color} or \usepackage{xcolor}
  basicstyle=\footnotesize\ttfamily, % the size of the fonts that are used for the code
  breakatwhitespace=false,         % sets if automatic breaks should only happen at whitespace
  breaklines=true,                 % sets automatic line breaking
  captionpos=none,                 % sets the caption-position to bottom
  aboveskip=\smallskipamount,
  belowskip=\smallskipamount,      % default is \medskipamount
  commentstyle=\color{mygreen},    % comment style
  deletekeywords={...},            % if you want to delete keywords from the given language
  escapeinside={\%*}{*)},          % if you want to add LaTeX within your code
  extendedchars=true,              % lets you use non-ASCII characters; for 8-bits encodings only, does not work with UTF-8
  frame=none,                      % adds a frame around the code (none, single)
  keepspaces=true,                 % keeps spaces in text, useful for keeping indentation of code (possibly needs columns=flexible)
  keywordstyle=\color{maroon},       % keyword style
  language=Java,                   % the language of the code
  morekeywords={*,...},            % if you want to add more keywords to the set
  numbers=none,                    % where to put the line-numbers; possible values are (none, left, right)
  rulecolor=\color{black},         % if not set, the frame-color may be changed on line-breaks within not-black text (e.g. comments (green here))
  showspaces=false,                % show spaces everywhere adding particular underscores; it overrides 'showstringspaces'
  showstringspaces=false,          % underline spaces within strings only
  showtabs=false,                  % show tabs within strings adding particular underscores
  stringstyle=\color{mymauve},     % string literal style
  tabsize=4,                       % sets default tabsize to 4 spaces
% title=\lstname                   % show the filename of files included with \lstinputlisting; also try caption instead of title
}
\lstset{style=javanonum}


% The 'subfigure' package does a nice job handling and labelling subfigures
\usepackage{subfig}

% If you want to use PostScript fonts (or the Nimbus knockoffs) try
%\usepackage{mathptmx}
%\usepackage[scaled=0.92]{helvet}

% The 'citesort' package puts mutliple citation numbers in order
%\usepackage{citesort}


% Generate intra-document links and also allows url and href commands for
% to generate embedded links in the document.
\usepackage{hyperref}
\hypersetup{
 	colorlinks=true,
 	allcolors=blue,
}

% Default placement of tables and figures is right after the text, then preferred at top of page
\usepackage{float}
\floatplacement{table}{ht}
\floatplacement{figure}{ht}