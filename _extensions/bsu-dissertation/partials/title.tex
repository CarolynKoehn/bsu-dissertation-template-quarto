%% TO-DO: define frontmatter commands %%

$if(title)$
\title{$title$$if(thanks)$\thanks{$thanks$}$endif$}
$endif$
$if(subtitle)$
$if(beamer)$
$else$
\usepackage{etoolbox}
\makeatletter
\providecommand{\subtitle}[1]{% add subtitle to \maketitle
  \apptocmd{\@title}{\par {\large #1 \par}}{}{}
}
\makeatother
$endif$
\subtitle{$subtitle$}
$endif$
\author{$for(authors)$$it.name.literal$$sep$ \and $endfor$}
\date{$date$}
$if(beamer)$
$if(institute)$
\institute{$for(institute)$$institute$$sep$ \and $endfor$}
$endif$
$if(titlegraphic)$
\titlegraphic{\includegraphics$if(titlegraphicoptions)$[$for(titlegraphicoptions)$$titlegraphicoptions$$sep$, $endfor$]$endif${$titlegraphic$}}$endif$
$if(logo)$
\logo{\includegraphics{$logo$}}
$endif$
$endif$


%%%%%%%%%%%%%%%%%%%%%%%%%%%%%%%%%%%%%%%%%%%%%%%%%%%%%%%%%%%%%%%%%%%%%%%%%%%%%%
% Front Matter Definitions
%%%%%%%%%%%%%%%%%%%%%%%%%%%%%%%%%%%%%%%%%%%%%%%%%%%%%%%%%%%%%%%%%%%%%%%%%%%%%%

% These definitions are used by the commands below to construct the front
% matter pages.

% Document title
$if(title)$
\title{$title$}
$endif$

% Author's name (must be exactly what name the Registrar has)
% normally it is of the form "First Middle Last"
\author{Wabi-sabi Admirer}

% Day of oral defense
\defenseDate{1st December, 2022}

% Month of graduation (do not abbreviate)
\graduationMonth{December}

% Year of graduation
\graduationYear{2022}

% Advisor (chair)
% Use full name, with surname last, without any titles
% The \advisor command accepts an optional argument that specifies the
% advisor's position, which defaults to "Advisor"; for example,
%
% \advisor[Chair]{Wabi-sabi the Great}
%
% causes "Chair" to appear on the approval page (the one with the signatures)
% in place of "Advisor"
\advisor{Wabi-sabi The Great, Ph.D.} % also the chair of your committee

% Committee members: \committeeA is listed first after the advisor,
% \committeeB second, etc.  An optional argument is accepted to replace
% the default "Committee Member" position.
\committeeA{Aaa, Ph.D.}
\committeeB{Bbb, Ph.D.}

% Masters or PhD committees normally have three members, but in case there is
% a need for more, use these commands
%\committeeC[Ex-Officio Committee Member]{ccc}
%\committeeD{ddd}

% Full name of the degree; normally "Master of Science"
%\degree{Master of Science}
\degree{Doctor of Philosophy}

% Full name of major
%\major{Computer Science}
\major{Computing}

% Major department
\department{Computer Science}

% College
\college{Engineering}

% Current department chair (at the moment, this is not used)
\departmentChair{Amit Jain}

% Document abstract
\abstract{ An \emph{abstract} is a brief summary of the document. A typical
  abstract provides a brief introduction, enough to provide context for the document, explains
  the purpose of the thesis or dissertation or project, and summarizes the major results and
  conclusions.  Keep in mind that a casual observer is likely to judge the content of the document
  by the abstract and title alone.  (There is an old adage: ``in a joke, the punchline comes
  at the end; in a paper [or thesis], it comes in the abstract.'')  A single concise paragraph
  usually suffices for the abstract.  If it spills onto a second page, it is probably too long.
}


% The remaining elements are optional

% \includeCopyright causes a copyright page to be produced.  It is not
% required; in fact, a copyright notice is no longer needed to ensure
% copyright protection.
\includeCopyright

% \maxPage is used to format the page numbers in the table of contents.
% The actual value is not important: the width of the value given is used
% as the maximum width of any page number.  The following values are suggested
%
% \maxPage{99} for less than 100 pages (this is the default)
% \maxPage{199} for less than 200 pages
% \maxPage{999} for less than 1000 pages
%
% (If the work is more than 1000 pages, consider cutting it down!)
% The space has to be wide enough for front matter page numbers, which are
% miniscule Roman numerals.  This might have to be widened if there are
% more than 12 front matter pages (which might happen if there is a long
% list of symbols or list of abbreviations, like this template has).
\maxPage{199}

% The Acknowledgments page is a place for the author to express thanks
% or generally acknowledge anyone or anything that assisted in the
% project or research.  If the student was supported even in part by
% funding from research grant, it needs to be acknowledged here.
% The Acknowledgments section can have multiple pages, just like the
% Abstract section, but like the abstract, more than one page is probably
% too long.  Acknowledgments are often written in the third person, but
% it is one place (the only place) in the document where first-person
% references (i.e., the use of "I") can be acceptable.
\acknowledgments{
The author wishes to express gratitude to Wabi-sabi.  This work would have been
partially supported by some particular grant, if there was one.
}

% The work can be "dedicated" to someone, typically family members or
% someone similar.  Unlike acknowledgments, which are included in almost
% every thesis or dissertation or project, many students omit the dedication.
\dedication{dedicated to Wabi-sabi}


% A brief autobiographical sketch can be included.  Many students omit this.
% If it is included, remember that it is a sketch, so keep it brief.  Also
% try to limit the content to academically relevant material, such as under-
% graduate work, previous graduate work, employment, etc.  Some students
% like to include their military service.  Otherwise, unless you have
% overcome some particular adversity, it's probably better to leave out
% personal information.
\biosketch{
Wabi-sabi Admirer was born admiring Wabi-sabi. Wabi-sabi Admirer has been tinkering
with admiration of Wabi-sabi for a long time. Now it is time to be blessed by
Wabi-sabi.
}