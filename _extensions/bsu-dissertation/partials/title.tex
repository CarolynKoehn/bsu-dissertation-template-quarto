%%%%%%%%%%%%%%%%%%%%%%%%%%%%%%%%%%%%%%%%%%%%%%%%%%%%%%%%%%%%%%%%%%%%%%%%%%%%%%
% Front Matter Definitions
%%%%%%%%%%%%%%%%%%%%%%%%%%%%%%%%%%%%%%%%%%%%%%%%%%%%%%%%%%%%%%%%%%%%%%%%%%%%%%

% These definitions are used by the commands below to construct the front
% matter pages.

% Document title
$if(title)$
\title{$title$}
$endif$

% Author's name (must be exactly what name the Registrar has)
% normally it is of the form "First Middle Last"
$if(author)$
\author{$author$}
$endif$

% Day of oral defense
$if(defenseDate)$
\defenseDate{$defenseDate$}
$endif$

% Month of graduation (do not abbreviate)
$if(graduationMonth)$
\graduationMonth{$graduationMonth$}
$endif$

% Year of graduation
$if(graduationYear)$
\graduationYear{$graduationYear$}
$endif$

% Advisor (chair)
% Use full name, with surname last, without any titles
% The \advisor command accepts an optional argument that specifies the
% advisor's position, which defaults to "Advisor"; for example,
%
% \advisor[Chair]{Wabi-sabi the Great}
%
% causes "Chair" to appear on the approval page (the one with the signatures)
% in place of "Advisor"
$if(advisor)$
\advisor$if(chair)$[Chair]$endif${$advisor$}
$endif$

% Committee members: \committeeA is listed first after the advisor,
% \committeeB second, etc.  An optional argument is accepted to replace
% the default "Committee Member" position.
$if(committeeA)$
\committeeA$if(committeeAtitle)$[$committeeAtitle$]$endif${$committeeA$}
$endif$
$if(committeeB)$
\committeeB$if(committeeBtitle)$[$committeeBtitle$]$endif${$committeeB$}
$endif$

% Masters or PhD committees normally have three members, but in case there is
% a need for more, use these commands
$if(committeeC)$
\committeeC$if(committeeCtitle)$[$committeeCtitle$]$endif${$committeeC$}
$endif$
$if(committeeD)$
\committeeD$if(committeeDtitle)$[$committeeDtitle$]$endif${$committeeD$}
$endif$

% Full name of the degree; normally "Master of Science"
%\degree{Master of Science}
$if(degree)$
\degree{$degree$}
$endif$

% Full name of major
$if(major)$
\major{$major$}
$endif$

% Major department
$if(department)$
\department{$department$}
$endif$

% College
$if(college)$
\college{$college$}
$endif$

% Current department chair (at the moment, this is not used)
$if(departmentChair)$
\departmentChair{$departmentChair$}
$endif$

% Document abstract
$if(abstract)$
\abstract{$abstract$}
$endif$


% The remaining elements are optional

% \includeCopyright causes a copyright page to be produced.  It is not
% required; in fact, a copyright notice is no longer needed to ensure
% copyright protection.
$if(includeCopyright)$
\includeCopyright
$endif$

% \maxPage is used to format the page numbers in the table of contents.
% The actual value is not important: the width of the value given is used
% as the maximum width of any page number.  The following values are suggested
%
% \maxPage{99} for less than 100 pages (this is the default)
% \maxPage{199} for less than 200 pages
% \maxPage{999} for less than 1000 pages
%
% (If the work is more than 1000 pages, consider cutting it down!)
% The space has to be wide enough for front matter page numbers, which are
% miniscule Roman numerals.  This might have to be widened if there are
% more than 12 front matter pages (which might happen if there is a long
% list of symbols or list of abbreviations, like this template has).
$if(maxPage)$
\maxPage{$maxPage$}
$endif$

% The Acknowledgments page is a place for the author to express thanks
% or generally acknowledge anyone or anything that assisted in the
% project or research.  If the student was supported even in part by
% funding from research grant, it needs to be acknowledged here.
% The Acknowledgments section can have multiple pages, just like the
% Abstract section, but like the abstract, more than one page is probably
% too long.  Acknowledgments are often written in the third person, but
% it is one place (the only place) in the document where first-person
% references (i.e., the use of "I") can be acceptable.
$if(acknowledgments)$
\acknowledgments{
$acknowledgments$
}
$endif$

% The work can be "dedicated" to someone, typically family members or
% someone similar.  Unlike acknowledgments, which are included in almost
% every thesis or dissertation or project, many students omit the dedication.
$if(dedication)$
\dedication{$dedication$}
$endif$


% A brief autobiographical sketch can be included.  Many students omit this.
% If it is included, remember that it is a sketch, so keep it brief.  Also
% try to limit the content to academically relevant material, such as under-
% graduate work, previous graduate work, employment, etc.  Some students
% like to include their military service.  Otherwise, unless you have
% overcome some particular adversity, it's probably better to leave out
% personal information.
$if(biosketch)$
\biosketch{$biosketch$}
$endif$